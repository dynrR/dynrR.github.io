\documentclass[12pt]{article}
\usepackage[english]{babel}
\usepackage[utf8]{inputenc}
\usepackage[T1]{fontenc}
\usepackage[textwidth=12cm,centering]{geometry}
\usepackage[x11names]{xcolor}
\usepackage{background}
\usepackage{hyperref}


\usepackage{lipsum}

%\newcommand\blfootnote[1]{%
%	\begingroup
%	\renewcommand\thefootnote{}\footnote{#1}%
%	\addtocounter{footnote}{-1}%
%	\endgroup
%}

\makeatletter
\def\blfootnote{\xdef\@thefnmark{}\@footnotetext}
\makeatother
\newcommand*\wb[3]{%
  {\fontsize{#1}{#2}\usefont{U}{webo}{xl}{n}#3}}
\definecolor{PennSky}{RGB}{0, 156, 222}
% The page frame
%\SetBgColor{Goldenrod3}
\SetBgColor{PennSky}
\SetBgAngle{0}
\SetBgScale{1}
\SetBgOpacity{1}
\SetBgContents{%
\begin{tikzpicture}
\node at (0.5\paperwidth,0) {\wb{80}{34}{A}\rule[60pt]{.2\textwidth}{0.4pt}%
  \raisebox{55pt}{%
  \makebox[.6\textwidth]{\ \fontsize{24}{29}\selectfont\scshape Dynr }}%
  \rule[60pt]{.2\textwidth}{0.4pt}\wb{80}{34}{B}};
\node at (2,-0.5\textheight) {\rule{0.4pt}{.8\textheight}};
\node at (19.5,-0.5\textheight) {\rule{0.4pt}{.8\textheight}};
\node at (0.5\paperwidth,-\textheight) {\wb{80}{34}{C}\rule[-10pt]{\textwidth}{0.4pt}\wb{80}{34}{D}} ;
\end{tikzpicture}%
}



% colorize text
%\newcommand*\ColText[1]{\textcolor{Goldenrod3}{#1}}
\newcommand*\ColText[1]{\textcolor{PennSky}{#1}}

% a tabular* for each food group
\newenvironment{Group}[1]
  {\noindent\begin{tabular*}{\textwidth}{@{}p{.8\linewidth}@{\extracolsep{\fill}}r@{}}
    {\fontsize{22}{29}\selectfont\ColText{#1}}\\[0.8em]}
  {\end{tabular*}}

% to format each entry
\newcommand*\Entry[2]{%
  \sffamily#1 & #2}

% to format each subentry
\newcommand*\Expl[1]{%
  \hspace*{1em}\footnotesize #1}

\pagestyle{empty}

\begin{document}
	\raisebox{20pt}{
  \makebox[.85\textwidth]{\ \fontsize{15}{29}\selectfont\scshape Dynamic Modeling in R }}\\
\begin{Group}{Getting Started}
\Entry{dynr.data()}{\href{https://rdrr.io/cran/dynr/man/dynr.data.html}{9.00}} \\
\Expl{Gather your data} \\
\Entry{getdx()}{\href{https://rdrr.io/cran/dynr/man/getdx.html}{9.95}} \\
\Expl{Explore and estimate smoothed derivative estimates} \\
\end{Group}

%\vfill
\vspace{0.25cm}

\begin{Group}{Preparing the Main Course}
\Entry{prep.measurement()}{\href{https://rdrr.io/cran/dynr/man/prep.measurement.html}{19.99}} \\ 
\Expl{This function can be used to prepare the measurement model} \\
\Entry{prep.matrixDynamics() or prep.formulaDynamics()}{\href{https://rdrr.io/cran/dynr/man/prep.formulaDynamics.html}{8.90}} \\ 
\Expl{These functions can be used to prepare the dynamic model} \\ 
\Entry{prep.initial()}{\href{https://rdrr.io/cran/dynr/man/prep.initial.html}{10.70}} \\ 
\Expl{This function is used to set initial conditions} \\
\Entry{prep.noise()}{\href{https://rdrr.io/cran/dynr/man/prep.noise.html}{10.70}} \\ 
\Expl{This function prepares the noise.. LOL} \\
\Entry{[\emph{Optional}] prep.regimes()}{\href{https://rdrr.io/cran/dynr/man/prep.regimes.html}{15.99}} \\ 
\Expl{Use this if your data has multiple regimes in it!} \\
\Entry{dynr.model()}{\href{https://rdrr.io/cran/dynr/man/dynr.model.html}{10.70}} \\ 
\Expl{Combine the previous steps; for use with $dynr.cook$()} \\
\Entry{dynr.cook()}{\href{https://rdrr.io/cran/dynr/man/dynr.cook.html}{29.99}} \\ 
\Expl{Our main course; use this to run all the previous functions.}\\
\Expl{Can be subbed with $dynr.mi$(), see below} \\
\Entry{[\emph{Alternative}] dynr.mi()}\href{https://rdrr.io/cran/dynr/man/dynr.mi.html}{{19.93}} \\ 
\Expl{Select this instead of $dynr.cook$() if your data are missing}\\
\Expl{This will return a $dynr.cook$() object and other goodies} \\
\end{Group}

%\fill
\vspace{0.25cm}

\begin{Group}{Desserts}
\Entry{summary()}{5.99} \\
\Entry{plot() or dynr.ggplot()}{\href{https://rdrr.io/cran/dynr/man/dynr.ggplot.html}{10.99}} \\
\Entry{plotFormula()}{\href{https://rdrr.io/cran/dynr/man/plotFormula.html}{19.99}} \\
\Entry{printex()}{\href{https://rdrr.io/cran/dynr/man/printex.html}{14.99}} \\
\end{Group}

\blfootnote{$^{*}$Prices are arbitrary and are only to keep the pun going. Click the prices to be redirected to a webpage containing more information on that function.}{}
\end{document}